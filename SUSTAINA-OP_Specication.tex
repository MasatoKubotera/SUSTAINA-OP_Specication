\documentclass[runningheads]{llncs}
\usepackage[T1]{fontenc}
\usepackage{graphicx}

\begin{document}

\title{CIT Brains\\Robot Specification for RoboCup2022}
\author{Masato Kubotera}

\institute{Chiba Institute of Technology, 2-17-1 Tsudanuma, Narashino, Chiba, JAPAN
\email{masatokubotera06@yahoo.co.jp}\\}

\maketitle

\begin{figure}
  \centering
  \includegraphics[width=0.8\textwidth]{fig/SUSTAINA-OP 6 bodies.eps}
  \end{figure}

\begin{table}[]
  \centering
  \begin{tabular}{ll}
    \hline
    Robot Name          & SUSTAINA-OP\\
    Height              & 646.61 mm\\
    Weight              & 5.18 kg\\
    Walking speed       & 0.4 m/s\\
    Degrees of freedom  & 19 DoF\\
    Actuators           & B3M-SC-1170-A (KONDO) x 10 pcs\\
                        & B3M-SC-1040-A (KONDO) x 9 pcs\\
    Sensors             & MPU-9250* (TDK InvenSense)\\
                        & \quad \tiny{* The included Magnetometer is not used}\\
                        & e-CAM50\_CUNX (e-con Systems)\\
    Computing unit      & Jetson Xavier NX 16GB (NVIDIA)\\
                        & EN715 (AVerMedia)\\                        
    Battery             & HP-G830C2800S3 LiPo 3S 2800mAh (Hyperion)\\
    \hline
  \end{tabular}
\end{table}

\renewcommand{\thefootnote}{\fnsymbol{footnote}}

\footnote[0]{Open platform for SUSTAINA-OP: \url{https://github.com/citbrains/SUSTAINA-OP}}
\footnote[0]{Team website: \url{http://www.cit-brains.net}}

\end{document}
